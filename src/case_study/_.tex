\section{Preliminary Case Study: Aralia Dataset}
\begin{frame}
    \Huge{\centerline{\textbf{Preliminary Case Study}}}
\end{frame}

\begin{frame}[t]{Overview: Aralia Dataset}
\begin{itemize}
  \item \textbf{Dataset Composition:} The Aralia collection consists of 43 trees.
  \item \textbf{Diverse Problem Sizes:} Small trees (e.g.\ 25--32 basic events) through mid-sized models with over 1{,}500 BEs.  
  \item \textbf{Wide Probability Range:} Top-event probabilities spanning from rare events near \(10^{-13}\) to fairly likely failures with probability above 0.7.  
  \item \textbf{Model Variability:} Some trees are primarily AND/OR, others incorporate more advanced gates (K/N, XOR, NOT), providing thorough coverage of typical (and atypical) fault tree logic structures.
\end{itemize}
\end{frame}

\subsection{Benchmarking Procedure}
\begin{frame}[t]
\frametitle{Benchmarking Setup: Hardware and Environment}
\begin{itemize}
  \item \textbf{Target Hardware:}
    \begin{itemize}
      \item GPU: NVIDIA\textsuperscript{\textregistered} GeForce GTX 1660 SUPER (6\,GB GDDR6, 1{,}408 CUDA cores).
      \item CPU: Intel\textsuperscript{\textregistered} Core\textsuperscript{TM} i7-10700 (2.90\,GHz, turbo-boost, hyperthreading).
    \end{itemize}
  \item \textbf{Software Stack:}
    \begin{itemize}
      \item SYCL-based (AdaptiveCpp/HipSYCL), with LLVM-IR JIT for kernel compilation.
      \item Compiler optimization at \texttt{-O3} for efficient code generation.
      \item Repeated runs (5+) to mitigate transient variations.
    \end{itemize}
  \item \textbf{Measured Time:} Includes entire wall-clock duration, from host-device transfers and JIT compilation to final result collection.
\end{itemize}
\end{frame}

\begin{frame}[t]
\frametitle{Monte Carlo Execution and Implementation}
\begin{itemize}
  \item \textbf{Objective:} Compute TOP event probabilities for all 43 trees.
  \item \textbf{Sampling Strategy:}
    \begin{itemize}
      \item Single pass per fault tree, generating as many samples as fit in 6\,GB GPU memory.  
      \item 128-bit Philox4x32x10 pseudo-random number generator, parallel threads.
    \end{itemize}
  \item \textbf{Bit-Packing Optimization:}
    \begin{itemize}
      \item Each group of 64 Monte Carlo outcomes stored in a single 64-bit word.  
      \item Enables vectorized instructions (e.g.\ \texttt{popcount}) and reduces memory I/O.
    \end{itemize}
  \item \textbf{Data Types:}
    \begin{itemize}
      \item Tallies in 64-bit integers.  
      \item Probability accumulations in double precision (64-bit float).  
    \end{itemize}
\end{itemize}
\end{frame}

\subsection{Performance \& Accuracy Benchmarks}
\subsection{Accuracy Benchmark: Relative error (Log-probability), Data-Parallel Monte Carlo vs Min-Cut Upper Bound and Rare-Event Approximation}
\begin{frame}
    \begin{figure}[p]
        \centering
        \includesvg[height=1\textheight]{src/figs/plots/rel_error.svg}
    \end{figure}
\end{frame}


\subsection{Performance Benchmark (Memory Consumption): Sampled Bits Per Event Per Iteration}
\begin{frame}
  \begin{multicols}{2}
    \begin{figure}[p]
        \centering
        \includesvg[height=1\textheight]{src/figs/plots/mem_allocation_lines_zoom_all.svg}
    \end{figure}
    \begin{figure}[p]
        \centering
        \includesvg[height=1\textheight]{src/figs/plots/mem_allocation_lines_lg.svg}
    \end{figure}
  \end{multicols}

\end{frame}



\section{Outlook}
\subsection{}
\begin{frame}
  \textbf{Limitations:}
  \begin{itemize}
      \item {Brute-force/naive Monte Carlo struggles when sampling rare-events.}
        \begin{itemize}
          \item{Implement importance sampling: WIP.}
        \end{itemize}
  \end{itemize}\par
  \vfill
  \textbf{Next Steps:}
  \begin{itemize}
      \item {Benchmark on larger (G-PWR, G-MHTGR) models.}
  \end{itemize}\par
  \vfill
  \textbf{Future Work:}
  \begin{itemize}
      \item {Embedding models from Knowledge Graph, for semantic representation.}
      \item {Gradient computation on Knowledge Graph.}
  \end{itemize}\par
\end{frame}